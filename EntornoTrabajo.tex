\section{Preparando el Entorno de Trabajo}
\label{intro}

Para desarrollar los contenidos de la asignatura, vamos a trabajar con ANTLR
\ifx\python\undefined
, Python
\fi
y Java en la IDE Visual Studio Code.  Los proyectos de software los gestionaremos con Maven y usaremos Git para versionado y GitHub como repositorio.

A continuación se explica brevemente como instalar las diferentes herramientas para Debian Linux.  Las configuraciones se probaron en Debian versiones 9, 10 y 11.

\textbf{Recomendación}: La instalación de paquetes se debe realizar con el \emph{superusario} o \verb|root|, ya sea haciendo \emph{login}, cambiando de usuario con \verb|su -| o utilizando \verb|sudo|. Sin embargo, por seguridad, es recomendable que la descarga manual de paquetes se realice con un usuario normal.
% En los comandos de consola, al \emph{superusario} lo denotaremos con comenzando las instrucciones con el símbolo \verb|#| y al usario normal con \verb|$|.

\ifx\python\undefined
\subsection{Python}
\label{python}

% VER --> PyPI y Anaconda como gestores de paquetes.  

Vamos a utilizar Python~3 para programar, que normalmente se encuentra instalado. Entonces. lo primero que debemos realizar es verificar la instalación mediante \verb|apt| como superusario:
\begin{lstlisting}[style=consola]
  apt install python3
\end{lstlisting}
En caso de necesitar instalarlo, aceptamos y esperamos que termine la ejecución.
\fi


\subsection{Java}
\label{Java}

Como herramienta base, vamos a necesitar el \emph{Java Developer Kit (JDK)} versión~11. Pueden descargarlo desde la página de \href{https://www.oracle.com/java/technologies/javase-downloads.html}{Oracle} o utilizar \href{https://openjdk.java.net/}{OpenJDK}, que normalmente se instala con el gestor de paquetes del sistema operativo.
\ifx\python\undefined
Necesitamos Java para poder ejecutar el \emph{plug--in} de ANTLR.
\fi
Realizamos la instalación mediante \verb|apt| como superusario:
\begin{lstlisting}[style=consola]
  apt install openjdk-11-jdk
\end{lstlisting}
Aceptamos y esperamos que termine la ejecución.

\subsection{Git}
\label{git}

Para gestionar el versionado y evolución del proyecto, vamos a utilizar la herramenta \href{https://git-scm.com/}{Git}.  Como repositorio en la nube el servicio de \href{https://github.com/}{GitHub}.
Realizamos la instalación mediante \verb|apt| como superusario:
\begin{lstlisting}[style=consola]
  apt install git
\end{lstlisting}
Aceptamos y esperamos que termine la ejecución.


\subsection{Visual Studio Code}
\label{vscode}

Como herramienta de desarrollo vamos a utilizar la IDE \href{https://code.visualstudio.com/}{Visual Studio Code}.  Para distribuciones Linux, es conveniente utilizar el gestor de paquetes apropiado (Ej. \href{https://wiki.debian.org/VisualStudioCode}{para Debian}). Luego, debemos instalar los \emph{plug--in} necesarios para trabajar:
\begin{enumerate}
	\ifx\python\undefined
	\item \hyperref[pluginPython]{Pyhon}
	\fi
	\item \hyperref[pluginJava]{Java}
	\item \hyperref[pluginMaven]{Maven}
	\item \hyperref[pluginANTLR]{ANTLR}
\end{enumerate}

\subsubsection{Instalación}
\label{vscodeInst}

En la página de Visual Studio Code se encuentran disponibles los instaladores para las plataformas soportadas.  El paquete Debian para arquitectura linux64 se puede descargar con el navegador (\href{https://code.visualstudio.com/docs/?dv=linux64_deb}{descargar}) o utilizando \verb|wget|.
Realizamos la descarga con \verb|wget| como usuario normal y la instalación mediante \verb|dpkg| como superusario:
\begin{lstlisting}[style=consola]
 wget https://code.visualstudio.com/sha/
            download\?build=stable\&os=linux-deb-x64
 dpkg -i code_XX_amd64.deb 
\end{lstlisting}
Aceptamos y esperamos que termine la ejecución.  \verb|XX| es el número de versión y \emph{snapshot}.  El comando \verb|wget| se muestra en dos líneas por limitación del texto, pero debe escribirse en un único renglón.

La instalación del paquete se realiza con \verb|dpkg| porque no está en los respositorios Debian.  Durante la instalación, se genera la entrada necesaria para que las actualizaciones se realicen directamente con \verb|apt|.


\ifx\python\undefined
\subsubsection{Python \emph{plug--in}}
\label{pluginPython}

El \emph{plug--in} \href{https://marketplace.visualstudio.com/items?itemName=ms-python.python}{Python} instala lo necesario para desarrollar y ejecutar Python.

La instalación se puede realizar de dos formas:
\begin{enumerate}
	\item con el atajo de teclado \verb|Ctl+Shift+X| o \emph{clickeando} el ícono \emph{Extensions} y buscándolo, o
    \item con el atajo de teclado \verb|Ctl+P| para ejecutar en el \emph{VS Code Quick Open} el comando
    \begin{verbatim}
		ext install ms-python.python
	\end{verbatim}
\end{enumerate}
\fi

\subsubsection{Java \emph{plug--in}}
\label{pluginJava}

El \emph{plug--in} \href{https://marketplace.visualstudio.com/items?itemName=vscjava.vscode-java-pack}{Java Extension Pack} instala todo lo necesario para trabajar con el lenguaje Java.

La instalación se puede realizar de dos formas:
\begin{enumerate}
	\item con el atajo de teclado \verb|Ctl+Shift+X| o \emph{clickeando} el ícono \emph{Extensions} y buscándolo, o
    \item con el atajo de teclado \verb|Ctl+P| para ejecutar en el \emph{VS Code Quick Open} el comando
    \begin{verbatim}
		ext install vscjava.vscode-java-pack1
	\end{verbatim}
\end{enumerate}


\subsubsection{Maven \emph{plug--in}}
\label{pluginMaven}

El \emph{plug--in} \href{https://marketplace.visualstudio.com/items?itemName=vscjava.vscode-maven}{Maven for Java} debería instalarse automáticamente al instalar el Java Extension Pack.  Si por algún motivo no se instaló, hacerlo manualmente.


\subsubsection{ANTLR \emph{plug--in}}
\label{pluginANTLR}

En la \href{https://www.antlr.org/tools.html}{página web de ANTLR} se pueden encontrar los \emph{plug--in} para diferentes IDEs.

\begin{figure}[t]
	\centering
	\includegraphics[width=3cm]{img/IconoANTLRvscode}
	\caption{ANTLR4 grammar syntax support -- Mike Lischke}
	\label{icono}
\end{figure}

Si bien en Visual Studio Code existen varias herramientas para ANTLR, vamos a utilizar el \emph{plug--in} de Mike Lischke \href{https://marketplace.visualstudio.com/items?itemName=mike-lischke.vscode-antlr4}{ANTLR4 grammar syntax support}~(Figura~\ref{icono}).

El \emph{plug--in} completo se encuentra publicado con acceso libre en GitHub.  Este documento se basa en la documentación del \href{https://github.com/mike-lischke/vscode-antlr4/tree/master/doc}{\emph{plug--in ANTLR}}.


La instalación se puede realizar de dos formas:
\begin{enumerate}
	\item con el atajo de teclado \verb|Ctl+Shift+X| o \emph{clickeando} el ícono \emph{Extensions} y buscándolo, o
    \item con el atajo de teclado \verb|Ctl+P| para ejecutar en el \emph{VS Code Quick Open} el comando
    \begin{verbatim}
		ext install mike-lischke.vscode-antlr4
	\end{verbatim}
\end{enumerate}

