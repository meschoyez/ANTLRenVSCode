\section{Preparando el Entorno de Trabajo}
\label{intro}

Para desarrollar los contenidos de la asignatura, vamos a trabajar con ANTLR y
\ifx\python\undefined
Python
\else
Java
\fi
en la IDE Visual Studio Code.  Los proyectos de software los gestionaremos con Maven y usaremos Git para versionado y GitHub como repositorio.

A continuación se explica brevemente como instalar las diferentes herramientas.  No hace falta instalar Git ya que las funcionalidades necesarias se encuentran disponibles en Visual Studio Code.

\ifx\python\undefined
\subsection{Python}
\label{python}

Vamos a utilizar Python~3 para programar y Anaconda como gestor de paquetes.  Los primero que debemos realizar es instalarlo\dots
\fi


\subsection{Java}
\label{Java}

Como herramienta base, vamos a necesitar el \emph{Java Developer Kit (JDK)} versión~11. Pueden descargarlo desde la página de \href{https://www.oracle.com/java/technologies/javase-downloads.html}{Oracle} o utilizar \href{https://openjdk.java.net/}{OpenJDK}, que normalmente se instala con el gestor de paquetes del sistema operativo.
\ifx\python\undefined
Necesitamos Java para poder ejecutar el \emph{plug--in} de ANTLR.
\fi

\subsection{Visual Studio Code}
\label{vscode}

Como herramienta de desarrollo vamos a utilizar la IDE \href{https://code.visualstudio.com/}{Visual Studio Code}.  Para distribuciones Lixux, es conveniente utilizar el gestor de paquetes apropiado (Ej. \href{https://wiki.debian.org/VisualStudioCode}{para Debian}).  Luego, debemos instalar los \emph{plug--in} necesarios para trabajar:
\begin{enumerate}
	\ifx\python\undefined
	\item \hyperref[pluginPython]{Pyhon}
	\fi
	\item \hyperref[pluginJava]{Java}
	\item \hyperref[pluginMaven]{Maven}
	\item \hyperref[pluginANTLR]{ANTLR}
\end{enumerate}


\ifx\python\undefined
\subsection{Python \emph{plug--in}}
\label{pluginPython}

El \emph{plug--in} \href{https://marketplace.visualstudio.com/items?itemName=ms-python.python}{Python} instala lo necesario para desarrollar y ejecutar Python.

\subsection*{Instalación del \emph{plug--in}}
\label{instalacionPython}

La instalación se puede realizar de dos formas:
\begin{enumerate}
	\item con el atajo de teclado \verb|Ctl+Shift+X| o \emph{clickeando} el ícono \emph{Extensions} y buscándolo, o
    \item con el atajo de teclado \verb|Ctl+P| para ejecutar en el \emph{VS Code Quick Open} el comando
    \begin{verbatim}
		ext install ms-python.python
	\end{verbatim}
\end{enumerate}
\fi

\subsection{Java \emph{plug--in}}
\label{pluginJava}

El \emph{plug--in} \href{https://marketplace.visualstudio.com/items?itemName=vscjava.vscode-java-pack}{Java Extension Pack} instala todo lo necesario para trabajar con el lenguaje Java.

\subsection*{Instalación del \emph{plug--in}}
\label{instalacionJava}

La instalación se puede realizar de dos formas:
\begin{enumerate}
	\item con el atajo de teclado \verb|Ctl+Shift+X| o \emph{clickeando} el ícono \emph{Extensions} y buscándolo, o
    \item con el atajo de teclado \verb|Ctl+P| para ejecutar en el \emph{VS Code Quick Open} el comando
    \begin{verbatim}
		ext install vscjava.vscode-java-pack1
	\end{verbatim}
\end{enumerate}


\subsection{Maven \emph{plug--in}}
\label{pluginMaven}

El \emph{plug--in} \href{https://marketplace.visualstudio.com/items?itemName=vscjava.vscode-maven}{Maven for Java} debería instalarse automáticamente al instalar el Java Extension Pack.  Si por algún motivo no se instaló, hacerlo manualmente.


\subsection{ANTLR \emph{plug--in}}
\label{pluginANTLR}

En la \href{https://www.antlr.org/tools.html}{página web de ANTLR} se pueden encontrar los \emph{plug--in} para diferentes IDEs.

\begin{figure}[b]
	\centering
	\includegraphics[width=3cm]{img/IconoANTLRvscode}
	\caption{ANTLR4 grammar syntax support -- Mike Lischke}
	\label{icono}
\end{figure}

Si bien en Visual Studio Code existen varias herramientas para ANTLR, vamos a utilizar el \emph{plug--in} de Mike Lischke \href{https://marketplace.visualstudio.com/items?itemName=mike-lischke.vscode-antlr4}{ANTLR4 grammar syntax support}~(Figura~\ref{icono}).

El \emph{plug--in} completo se encuentra publicado con acceso libre en GitHub.  Este documento se basa en la documentación del \href{https://github.com/mike-lischke/vscode-antlr4/tree/master/doc}{\emph{plug--in ANTLR}}.


\subsection*{Instalación del \emph{plug--in}}
\label{instalacionANTLR}

La instalación se puede realizar de dos formas:
\begin{enumerate}
	\item con el atajo de teclado \verb|Ctl+Shift+X| o \emph{clickeando} el ícono \emph{Extensions} y buscándolo, o
    \item con el atajo de teclado \verb|Ctl+P| para ejecutar en el \emph{VS Code Quick Open} el comando
    \begin{verbatim}
		ext install mike-lischke.vscode-antlr4
	\end{verbatim}
\end{enumerate}

