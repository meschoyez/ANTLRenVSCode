\documentclass[a5paper,10pt]{article}

%\usepackage{ebook}

\usepackage[spanish]{babel}
\usepackage[utf8]{inputenc}
\usepackage[T1]{fontenc}
\usepackage[]{times}
\usepackage[colorlinks=true]{hyperref}

\author{Maximiliano A. Eschoyez}
\title{ANTLR en Visual Studio Code}
\date{Mayo 2019}

\begin{document}
%\ebook
\maketitle

\begin{abstract}
	Esta guía tiene como fin el uso de ANTLR en la IDE Visual Studio Code. Se explica desde la instalación hasta la generación de los diferentes gráficos.
\end{abstract}

\section{Instalación del \emph{plug--in}}

En la \href{https://www.antlr.org/tools.html}{página web de ANTLR} se pueden encontrar los \emph{plug--in} para diferentes IDEs.

Si bien para Visual Studio Code existen más herramientas para ANTLR, vamos a utilizar el \emph{plug--in} de Mike Lischke \href{https://marketplace.visualstudio.com/items?itemName=mike-lischke.vscode-antlr4}{ANTLR4 grammar syntax support}.

La instalación se puede realizar de dos formas:
\begin{enumerate}
	\item con el atajo de teclado \verb|Ctl+Shift+x| o \emph{clickeando} el ícono \emph{Extensions} y buscándolo, o
    \item con el atajo de teclado \verb|Ctl+p| para ejecutar en el \emph{VS Code Quick Open} el comando
    \begin{verbatim}
	      ext install mike-lischke.vscode-antlr4
	\end{verbatim}.
\end{enumerate}


\section{Cosas que puedo hacer}



\end{document}
