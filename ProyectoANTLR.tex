\section{Proyecto con ANTLR}
\label{proyecto_antlr}

Con el proyecto Java creado y listo para trabajar, vamos a realizar las configuraciones necesarias y luego crear un archivo  simple para ANTLR.

\subsection{Configuración de ANTLR}
\label{conf_antlr}

Para que la generación de los archivos y \emph{debugging} responda a los requerimientos, es encesario realizar las configuraciones necesarias.

\lstinputlisting[float,style=miXML,caption={Sección ANTLR de \texttt{launch.json}.},label=launchANTLR]{code/launchJsonAntlr.json}

En el Código~\ref{launchANTLR} se muestra la parte del archivo \verb|settings.json| con la configuración del \emph{plug--in} de ANTLR para que se generen automáticamente los archivos correspondientes a los analizadores léxico y sintáctico y los \emph{listeners} y \emph{visitor}.  Estos archivos se vuelven a generar cada vez que se guarda el archivo ANTLR, por lo cual, no deben modificarse.  Las entradas a configurar y su significado son:
\begin{itemize}
	\item \verb|name|, \verb|type| y \verb|request| agregan ANTLR a las opciones de depuración,
	\item \verb|grammar| indica el archivo ANTLR a usar, siendo \verb|${file}| el archivo que tenemos abierto (hacer foco y luego usar \verb|F5|),
	\item \verb|input| es el archivo a interpretar (ajustar según necesidad),
	\item \verb|startRule| es la regla sintáctico o \emph{símbolo} inicial (ajustar según necesidad),
	\item \verb|printParseTree| y \verb|visualParseTree| se usan para indicar la generación o no del árbol sintáctico en modo texto y gráfico, respectivamente.
\end{itemize}

\ifx\python\undefined
\lstinputlisting[float,style=miXML,caption={Sección ANTLR de \texttt{settings.json}.},label=settingsANTLR]{code/settingsPython.json}
\else
\lstinputlisting[float,style=miXML,caption={Sección ANTLR de \texttt{launch.json}.},label=settingsANTLR]{code/settingsJava.json}
\fi

Para poder analizar en detalle las reglas, el comportamiento y la ejecución del analizador léxico y el analizador sintáctico se debe configurar el archivo \verb|settings.json| como se muestra en el Código~\ref{settingsANTLR}.  La sección \verb|antlr4.generation| configura la ejecución de ANTLR para la genera de archivos que, por simplicidad, se recomienda usarla así.  La sección \verb|antlr4.debug| setea la vista inicial durante la depuración y puede cambiarse durante la visualización.



\subsection{Archivos de ANTLR}
\label{archivo_antlr}

Los archivos de ANTLR llevan extensión \verb|.g| o \verb|.g4|, pero utilizaremos la segunda opción.  El atajo de teclado para crear un archivo vacío es \verb|Ctl+N|, que para hacer efectivo el coloreo hay que guardarlo (\verb|Ctl+S|) con la extensión apropiada.

ANTLR permite la generación del \emph{lexer} y del \emph{parser}.  Por lo tanto, los archivos \verb|.g4| pueden ser para el primero, el segundo o ambos combinados.  Nosotros utilizaremos archivos combinados dentro del paquete que contiene el método \verb|main()| para facilitar la ejecución y visualización de resultados.  En particular, en el proyecto ejemplo guardaremos el archivo \verb|.g4| en la carpeta \verb|src/main/java|.


\subsection{Ejemplo Archivo ANTLR}
\label{ejemplo_archivo_antlr}

A modo de ejemplo, el Código~\ref{id_g4} es un archivo \verb|.g4| del cual se generará el \emph{lexer} y el \emph{parser} que realizan la búsqueda de identificadores tipo Java (nombres de variable, clase o método).  Tanto las reglas léxicas como las gramaticales comienzan con un identificador o etiqueta y terminan en punto y coma (\verb|;|).  Los dos puntos (\verb|:|) indican el comienzo de la regla y la barra vertical o \emph{pipe} (\verb-|-) separan las distintas reglas alternativas.  Las expresiones regulares para la detección de \emph{tokens} se etiquetan con nombres en mayúsculas, como ser \verb|ID| en la línea~10.  Las reglas gramaticales se etiquetan con nombres en minúsculas, como ser \verb|s|.

\lstinputlisting[float,style=miANTLR,caption={Ejemplo de archivo \texttt{.g4}.},label=id_g4]{code/id.g4}

Las palabras reservadas del Código~\ref{id_g4} significan lo siguiente:
\begin{description}
	\item[\texttt{grammar}] Al comienzo del archivo (línea~1) se indica qué queremos generar, siendo \verb|grammar| la palabra reservada tanto para un \emph{parser} como para un archivo combinado. La otra alternativa sería \verb|lexer|.  La palabra \verb|id| es el nombre del \emph{parser}.
	\item[\texttt{@header}] En la línea~3 se utiliza el bloque indicado como \verb|@header| para colocar código fuente en Java que necesitamos que aparezca en todos los código fuente generados por ANTLR.  En el ejemplo se consigna solamente el \verb|package| al que pertenecen.
	\item[\texttt{fragment}] En las líneas~7 y~8 la palabra \verb|fragment| indica que la expresión regular se utilizará para construir expresiones regulares más complejas, por lo tanto, no se utiliza durante el análisis léxico.
\end{description}

Cada vez que se grabe el archivo en disco, el \emph{plug--in} de ANTLR regenerará todos los archivos.

