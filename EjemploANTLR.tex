\section{Utilizando ANTLR en el proyecto}
\label{archivo_antlr}

Con el proyecto Java creado y listo para trabajar, vamos a crear un archivo para ANTLR simple. Los archivos de ANTLR llevan extensión \verb|.g| o \verb|.g4|, pero utilizaremos la segunda opción.  El atajo de teclado para crear un archivo vacío es \verb|Ctl+N|, que para hacer efectivo el coloreo hay que guardarlo (\verb|Ctl+S|) con la extensión apropiada.

ANTLR permite la generación del \emph{lexer} y del \emph{parser}.  Por lo tanto, los archivos \verb|.g4| pueden ser para el primero, el segundo o ambos combinados.  Nosotros utilizaremos archivos combinados dentro del paquete que contiene el método \verb|main()| para facilitar la ejecución y visualización de resultados.  En particular, en el proyecto ejemplo guardaremos el archivo \verb|.g4| en la carpeta \verb|PrimerProyecto| (dentro de \verb|src/main/java|).


\subsection{Ejemplo Archivo ANTLR}
\label{ejemplo_archivo_antlr}

A modo de ejemplo, el Código~\ref{id_g4} es un archivo \verb|.g4| del cual se generará el \emph{lexer} y el \emph{parser} que realizan la búsqueda de identificadores tipo Java (nombres de variable, clase o método).  Tanto las reglas léxicas como las gramaticales comienzan con un identificador o etiqueta y terminan en punto y coma (\verb|;|).  Los dos puntos (\verb|:|) indican el comienzo de la regla y la barra vertical o \emph{pipe} (\verb-|-) separan las distintas reglas alternativas.  Las expresiones regulares para la detección de \emph{tokens} se etiquetan con nombres en mayúsculas, como ser \verb|ID| en la línea~10.  Las reglas gramaticales se etiquetan con nombres en minúsculas, como ser \verb|s|.

\lstinputlisting[float,style=miANTLR,caption={Ejemplo de archivo \texttt{.g4}.},label=id_g4]{code/id.g4}

Las palabras reservadas del Código~\ref{id_g4} significan lo siguiente:
\begin{description}
	\item[\texttt{grammar}] Al comienzo del archivo (línea~1) se indica qué queremos generar, siendo \verb|grammar| la palabra reservada tanto para un \emph{parser} como para un archivo combinado. La otra alternativa sería \verb|lexer|.  La palabra \verb|id| es el nombre del \emph{parser}.
	\item[\texttt{@header}] En la línea~3 se utiliza el bloque indicado como \verb|@header| para colocar código fuente en Java que necesitamos que aparezca en todos los código fuente generados por ANTLR.  En el ejemplo se consigna solamente el \verb|package| al que pertenecen.
	\item[\texttt{fragment}] En las líneas~7 y~8 la palabra \verb|fragment| indica que la expresión regular se utilizará para construir expresiones regulares más complejas, por lo tanto, no se utiliza durante el análisis léxico.
\end{description}

Cada vez que se grabe el archivo en disco, el \emph{plug--in} de ANTLR regenerará todos los archivos.

